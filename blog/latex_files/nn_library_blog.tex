\documentclass{article}
\usepackage{amsmath}
\usepackage{hyperref}
\usepackage{pythonhighlight}
\setlength\parindent{0pt}

\begin{document}

\textbf{Introduction} \\

After the 'Applied Machine Learning' course in Taltech, I became interested in
neural networks and how they work. In the course we were building a few basic networks 
in Keras so that we would get some kind of introduction to machine learning without 
diving into details.\\

Even though the course was very interesting, I felt that I wanted to know more about 
how things work 'under the hood' and decided to build (or at least try to build) a neural network library
using only Python and Numpy with the intention of learning about the math and algorithms behind the networks in the
progress.\\

The idea is to build this library without setting any specific targets except for learning
something new and form an understanding about how those networks work. This document is intended 
to be a kind of blog in which I will write about the progress I have made.\\

\textbf{Building the base for the project} \\

To start out, I found a very nice tutorial:\\
\href{https://towardsdatascience.com/math-neural-network-from-scratch-in-python-d6da9f29ce65}{Simple neural networks tutorial} \\
For understanding backpropagation, Andrej Karpathy's tutorials were incredibly helpful: \\
\href{https://www.youtube.com/watch?v=VMj-3S1tku0&list=PLAqhIrjkxbuWI23v9cThsA9GvCAUhRvKZ}{Andrej Karpathy's series on neural networks} \\

Following the first tutorial, I built a (very very) basic library that implements a fully connected layer, Tanh function and 
Mean Squared Error function. The plan is to modify this base project and add new things on top of it. \\

\textbf{First modifications} \\

I decided to modify the prject right away. \\

In the original implementation the user would have to add both the activation function and the corresponding derivative 
function as arguments when creating a new Activation layer. I thought that it would be better if the user could just add 
the activation function as a string and the program would define the necessary functions itself. \\

Instead of this...:
\begin{python}
    net.add(ActivationLayer(tanh, tanh_prime))
\end{python}
...an activation layer can be added like this:
\begin{python}
    net.add(Activation('tanh'))
\end{python}

I also added a Sigmoid activation function and a ReLU activation function. \\

\textbf{Attaching activation functions to the layers} \\

Activation functions are now attached to the layers. When defining a layer, you have to pass an activation function as an
argument. At the moment it's not possible to create a layer without an activation function but it will be at some point. \\

\textbf{Changes in weight initialization}

The weights for fully connected layers are initialized as a normal distribution and biases as zeros. I would have to read about 
weight initialization techniques in the future but for now even this small change made a difference. \\

\textbf{Creating models functionally}

Instead of adding layers one by one like this...:
\begin{python}
net = Network()
net.add(FCLayer(2, 3))
net.add(ActivationLayer(tanh, tanh_prime))
net.add(FCLayer(3, 1))
net.add(ActivationLayer(tanh, tanh_prime))
\end{python}

...I liked the idea of adding them functionally, like this:
\begin{python}
inputs = Dense(input_size=2, output_size=32, activation='tanh')
x = Dense(output_size=16, activation='relu')(inputs)
x = Dense(output_size=8, activation='relu')(x)
x = Dense(output_size=4, activation='relu')(x)
outputs = Dense(output_size=1, activation='sigmoid')(x)

model = Model(inputs, outputs)
\end{python}

After the layers are defined, the model is compiled and a layer-graph is created from inputs to outputs. 
I thought that maybe this approach opens up the possibility to create non-sequential graphs in the future. \\

\textbf{Adding loss functions on model compilation} \\

Instead of adding loss functions like this...:
\begin{python}
model.use_loss_function('mse')
\end{python}

...I changed it so that the loss function is defined when the model is compiled:
\begin{python}
model.compile(loss_fn='mse')
\end{python}

I also made a separate class for a loss function and created forward and backward methods for it. I thought 
that this way it would be more consistent with the rest of the code. \\

Besides changing loss functions, I organized files into their folders so that in the future it would be easier
to find what's needed.

\textbf{Categorical Cross-Entropy loss and Softmax activation function} \\

I implemented Categorical Cross-Entropy and a Softmax activation function. So far, it was the 
most difficult thing to do. I was trying to figure out the respective derivative functions for ages 
and in the end, failed. Then I discovered that it's a common practice to calculate the derivatives of 
those functions in tandem, in which case the derivative function is very simple and fast. \\

Because those functions are mostly used together anyway, I decided to make it possible to use them 
only in combination with each other, write a function that calculates their combined derivative and 
just move on for now. I might (should?) come back to it at some point in the future (when my math 
has improved, hopefully). \\

\textbf{Testing the project using Iris dataset} \\

Until now I had tested the network only with the basic truth table dataset consisting of 4 examples
to see if it learns at all, so it was quite exciting to test it with something slightly bigger. I used 
the Iris dataset which consists of 150 examples each of which has 4 features and is classified 
into 1 of 3 possible classes. \\

I created a couple of temporary pre-processing functions - to one-hot encode the labels 
and to divide the dataset into a train set and a test set. The train set had 70\% of the
samples and the test set 30\%. \\

The model consisted of 3 layers
\begin{itemize}
    \item Dense layer with 4 inputs (features) and 8 outputs using Tanh 
    \item Dense layer with 8 outputs using ReLU
    \item Dense layer with 3 (possible classes) outputs using Softmax
\end{itemize}

The model used Categorical Cross-Entroy loss. \\

I was playing around with different learning rates and discovered that 
0.01 worked the best. 500 epochs were used for training. \\

I trained the model using the training set and then let it predict using the test set. I did this for
5 times. Each time the train/test split was different as they were created randomly. I got the following results when predicting:
\begin{itemize}
    \item 45/45 - 100\%
    \item 39/45 - 86.67\%
    \item 45/45 - 100\%
    \item 44/45 - 97.78\%
    \item 45/45 - 100\%
\end{itemize}

I'm rather happy with the results, somehow I was expecting worse. It's a good place to 
move on from. \\
I will write the functions that I used to pre-process the data here as well. \\

The function for importing the Iris dataset:
\begin{python}
def import_data(filename: str) -> pd.DataFrame:
    names = [
        "sepal_length",
        "sepal_width",
        "petal_length",
        "petal_width",
        "class",
    ]
    data = pd.read_csv(filename, names=names)
    data['class'] = pd.Categorical(data['class'])
    data['class'] = data['class'].cat.codes

    return data
\end{python}

The function for one-hot encoding the labels:
\begin{python}
def hot_encode(y, n_classes):
    hot_encoded = np.zeros((len(y), n_classes))
    for i, c in enumerate(y):
        hot_encoded[i][c] = 1
    return hot_encoded
\end{python}

The function for separating the data into train/test splits:
\begin{python}   
def train_test_split(data: pd.DataFrame, split: float) -> np.ndarray:
    X = data.drop('class', axis='columns').to_numpy()
    y = data['class'].to_numpy()
    y = hot_encode(y, 3)

    shuffle = np.arange(len(data))
    random.shuffle(shuffle)
    n_split = int(len(data) * split)

    X_train = X[shuffle[:n_split]]
    y_train = y[shuffle[:n_split]]
    X_test = X[shuffle[n_split:]]
    y_test = y[shuffle[n_split:]]

    X_train = np.reshape(X_train, (X_train.shape[0], 1, X_train.shape[-1]))
    y_train = np.reshape(y_train, (y_train.shape[0], 1, y_train.shape[-1]))
    X_test = np.reshape(X_test, (X_test.shape[0], 1, X_test.shape[-1]))
    y_test = np.reshape(y_test, (y_test.shape[0], 1, y_test.shape[-1]))

    return X_train, y_train, X_test, y_test
\end{python}

The function for building the model (for some reason in this branch the Dense layers were still called Linear layers - I have already changed the name):
\begin{python}
def build_model():
    inputs = Linear(input_size=4, output_size=8, activation='tanh')
    x = Linear(output_size=8, activation='relu')(inputs)
    outputs = Linear(output_size=3, activation='softmax')(x)

    model = Model(inputs, outputs)
    return model
\end{python}

The function for calculating the accuracy:
\begin{python}
def calculate_accuracy(preds, y_test, print_comparison=False):
    correct = 0
    for pred, true in zip(preds, y_test):
        y_pred = np.argmax(pred)
        y_true = np.argmax(true)
        if y_pred == y_true:
            correct += 1
        if print_comparison:
            print(y_pred, y_true)
    print()
    print(f"Predicted correctly {correct}/{len(preds)} which" +
          f"is {round(correct / len(preds) * 100, 2)}%")
\end{python}

Function calls and training-testing the model:
\begin{python}
data = import_data('iris.data')
split = 0.7
X_train, y_train, X_test, y_test = train_test_split(data, split)

model = build_model()
model.compile(loss_fn='categorical_cross_entropy')
model.fit(X_train,
          y_train,
          epochs=500,
          learning_rate=0.01,
          print_loss=True)

preds = model.predict(X_test)
accuracy = calculate_accuracy(preds, y_test)
\end{python}

\end{document}